\newcommand{\imagen}[4][0.8]{%[0.8] parámetro por defecto para la escala.
	\begin{figure}[H]\centering
	\includegraphics[width=#1\textwidth]{img/#2}
	\caption{#4}{#3}
	\end{figure}
	}
	%EJEMPLOS
	%"\imagen{007}{\label{fig:007}}{Hola}". Imagen->img/007, con label "fig:007" y descripcion Hola. Escala 0.8.
	%"\imagen[0.5]{007}{fig:007}{Hola}". Mismo anterior, con escala forzada a [0.5] del texto.
	
\newcommand{\bimagen}[6]{%[1] parámetro por defecto para la escala de la minipage.
	\hspace{-0.04\textwidth}
	\begin{minipage}[t]{0.47\textwidth}
	\begin{figure}[H]\centering
	\includegraphics[width=1\textwidth]{img/#1}
	\caption{#3}{#2}
	\end{figure}
	\end{minipage}
	\hfill
	\begin{minipage}[t]{0.47\textwidth}
	\begin{figure}[H]\centering
	\includegraphics[width=1\textwidth]{img/#4}
	\caption{#6}{#5}
	\end{figure}
	\end{minipage}
	\vspace{1em}
}
	
\newcommand{\trimagen}[9]{%[1] parámetro por defecto para la escala de la minipage.
	\hspace{-0.05\textwidth}
	\begin{minipage}[t]{0.3\textwidth}
	\begin{figure}[H]\centering
	\includegraphics[width=1\textwidth]{img/#1}
	\caption{#3}{#2}
	\end{figure}
	\end{minipage}
	\hspace{0.05\textwidth}
	\begin{minipage}[t]{0.3\textwidth}
	\begin{figure}[H]\centering
	\includegraphics[width=1\textwidth]{img/#4}
	\caption{#6}{#5}
	\end{figure}
	\end{minipage}
	\hspace{0.05\textwidth}
	\begin{minipage}[t]{0.3\textwidth}
	\begin{figure}[H]\centering
	\includegraphics[width=1\textwidth]{img/#7}
	\caption{#9}{#8}
	\end{figure}
	\end{minipage}
	\vspace{1em}
}
%EJEMPLO
%\trimagen
%	{imagen1}{\label{fig:1}}
%		{Descripcion1}
%	{imagen2}{\label{fig:2}}
%		{Descripcion2}
%	{imagen3}{\label{fig:3}}
%		{Descripcion3}